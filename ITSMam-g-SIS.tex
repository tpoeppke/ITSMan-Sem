\documentclass{semdoc}
% Template: $Id: t01_txt.tex,v 1.7 2000/05/23 12:13:37 bless Exp $
% -----------------------------------------------------------------------------
% Seminarbeitrag
% -----------------------------------------------------------------------------
% Kommentare beginnen mit einem %-Zeichen
\docbegin
% --> Oberhalb der Linie bitte nichts ändern.
% ---------------------------------------------------------------------------
% \/ \/ \/ \/ \/ \/ \/ \/ \/ \/ \/ \/ \/ \/ \/ \/ \/ \/ \/ \/ \/ \/ \/ \/ \/ 
% Stellen, an denen etwas geaendert werden soll, sind wie hier gekennzeichnet.
% /\ /\ /\ /\ /\ /\ /\ /\ /\ /\ /\ /\ /\ /\ /\ /\ /\ /\ /\ /\ /\ /\ /\ /\ /\ 

%
% ---------------------------------------------------------------------------
% \/ \/ \/ \/ \/ \/ \/ \/ \/ \/ \/ \/ \/ \/ \/ \/ \/ \/ \/ \/ \/ \/ \/ \/ \/ 
% --> Bitte den Titel des Beitrages in die nächste Zeile eintragen:
\title{IT-Sicherheitsmanagement Seminar: Group-Centric Models for Secure Information Sharing (g-SIS)}

\newtheorem{definition} [Definition]%
% --> ... und den Namen des Autors:
\author{Uwe Kühn, Tobias Pöppke}
% --> optional eine URL mailto:... oder http://www...
%\authorURL{mailto:tobias.poeppke@student.kit.edu}
% /\ /\ /\ /\ /\ /\ /\ /\ /\ /\ /\ /\ /\ /\ /\ /\ /\ /\ /\ /\ /\ /\ /\ /\ /\ 
% -----------------------------------------------------------------------------
%
%
\maketitle
%
% -----------------------------------------------------------------------------
% ACHTUNG - ACHTUNG - ACHTUNG - ACHTUNG - ACHTUNG - ACHTUNG - ACHTUNG - ACHTUNG
% -----------------------------------------------------------------------------
% --> Im Text sollte \section der "höchste" Gliederungsbefehl sein,
%     also bitte kein \chapter oder gar \part verwenden.
%     Der Text kann aber mit \section{}, \subsection{}, \subsubsection{}
%     untergliedert werden.
%
%     Bitte nicht die Befehle \input oder \include verwenden, also den
%     Text _nicht_ in mehrere Dateien aufteilen! 
%
%     \newpage oder manuelle Zeilenumbrüche (\\) sollten ebenfalls nicht 
%     verwendet werden.
%
%     Bitte darauf achten, dass im _Quelltext_ ein Abschnitt nicht nur 
%     in einer Zeile steht (das macht z.B. Word beim Exportieren ohne
%     Zeilenumbruch), der Abschnitt sollte auch bei 80 Zeichen pro Zeile
%     noch lesbar sein, d.h. Zeilenumbrüche im Quelltext bitte entsprechend
%     einfügen.
%
%     Referenzen auf andere Abschnitte sind bitte mit \ref{...}, wie 
%     anschliessend gezeigt, anzugeben und nicht etwa als Text wie
%     "siehe auch Abschnitt 2.2"
%
%     Anführungszeichen sind nicht einfach so "Text" einzugeben, sondern so:
%     "`Text"', andernfalls gibt es Fehler.
%
%     Das Zeichen ~ erzeugt einen Leerraum an dem aber nicht getrennt wird.
%     Weitere Trennstellen:
%     "- = zusätzliche Trennstelle
%     "| = Vermeidung von Ligaturen und mögliche Trennung (bsp: Schaf"|fell)
%     "~ = Bindestrich an dem keine Trennung erlaubt ist (bsp: bergauf und "~ab)
%     "= = Bindestrich bei dem Worte vor und dahinter getrennt werden dürfen
%     "" = Trennstelle ohne Erzeugung eines Trennstrichs (bsp: und/""oder)
%
%     Weiterer Hinweis: 
%     KEIN Glossar erstellen! Abkürzungen sind im Text zu erklären!
%     Der Text ist mit z.B. mit ispell auf Schreibfehler zu prüfen 
%     (am besten aus dem Emacs heraus, 
%     Menü: Edit -> Spell -> Select Deutsch8, dann Edit -> Spell -> Check Buffer)
%
% ---------------------------------------------------------------------------
% \/ \/ \/ \/ \/ \/ \/ \/ \/ \/ \/ \/ \/ \/ \/ \/ \/ \/ \/ \/ \/ \/ \/ \/ \/ 
% --> ... und hier kommt der Text: 
\section{Einleitung} % in die Klammern die Ueberschrift 
 \label{Introduction} % und ein Label
 
TODO: Subscription Service, Meeting room Metaphern.
\cite{gsis1} \cite{gsis2} 
 
 
% % Das Label ist eine Marke, auf die man später mit \ref{labelname} verweisen kann.
% % BITTE ALLE LABELS MIT t01_ BEGINNEN.
% % Hier den Text eintragen:
% Ich bin der Beispieltext und habe eine Referenz auf Abschnitt~\ref{tmp_Introduction}. 
% Verschiedene Schriftgrößen wie {\small small}, {\footnotesize footnotesize} und
% Stile \emph{hervorgehobener Text}, \textit{kursiv}, \textsl{schräggestellt} (anders
% als kursiv), \textbf{Fettdruck} (sollte \emph{nicht} verwendet werden).
% Text in Anführungszeichen sieht so aus: "`korrekt"'.

% Und nun ein Bild:
% % Hinweis: der Text in der Abbildung sollte nicht größer als die
% % Schriftgröße des Fließtextes (11pt) sein und möglichst mit 
% % serifenlosen Font (Helvetica, Arial).
% %
% \begin{figure}[htb] 
	% \centerline{\includegraphics*[width=0.8\textwidth]{an_example_picture}}
% \caption{Dies ist die Bildunterschrift} 
% \label{tmp_Bild1}
% \end{figure}

% Umlaute gibt es auch: ä, ö, ü, Ä, Ö, Ü, ß. Alternativ kann man notfalls 
% auch "a, "o, "o, "A, "O, "U, "s eingeben. Die Umlaute werden nach ISO 8859-1
% codiert. Eingabemöglichkeiten im \texttt{emacs} wie folgt:
% \begin{enumerate}
% \item
% Wenn keine Umlaute auf der Tastatur vorhanden sind, kann man 
% z.B. mit folgender Tastenkombination (Tasten nacheinander drücken)
% ein ä erzeugen: \fbox{\parbox{1.05cm}{\tiny compose\\ character}}
% dann \fbox{\"{}} dann \fbox{a}.
% % Im Klartext: [compose character] ["] [a]
% \item
% Einfacher geht's jedoch mit dem \emph{iso-accents-mode} des
% \texttt{emacs}: Einfach \fbox{\"{}} gefolgt von \fbox{a}
% drücken.
% % Im Klartext: ["] [a]
% \end{enumerate}

% % beachte Anweisungen am Ende des Dokumentes, wenn Literatur referenziert
% % werden soll
% Nun einen Verweis auf eine Literaturstelle: \cite{t01_buch1}.
% Die Literatur ist in der BIB\TeX-Datei einzugeben (Endung \texttt{.bib}).
% Es wird letztlich nur die Literatur aus der Datei aufgeführt, die auch im Text referenziert wurde. Möchte man dies ändern, so muss die unten stehende Zeile '\verb|%\nocite{*}|' auskommentiert werden. Der Editor
% \texttt{emacs} unterstützt die Eingabe im BIB\TeX-Format (nach
% Laden einer Datei mit Endung \texttt{.bib}).


% Als nächstes präsentieren wir eine Aufzählung:
% \begin{itemize}
% \item Punkt1\\
      % Eine Formel (nicht numeriert): 
      % \[ \frac{1}{n} \cdot \sum_{i=1}^{n}{(\overline{x}-x_i)^2} \]
% \item Punkt2
% \item Punkt3
% \end{itemize}

% Man kann übrigens auch Bilder referenzieren: Bild~\ref{tmp_Bild1} oder
% Unterabschnitte wie z.B. den nächsten Unterabschnitt~\ref{tmp_Subsection}.
% URLs lassen sich so angeben: \urltext{http://www.uni-karlsruhe.de/} oder
% explizit \href{http://www.uni-karlsruhe.de/}{Universität Karlsruhe (TH)}
% (dann erscheint die URL nicht als zwangsweise im Text).

\section{Grundlagen}
\label{basics}

Policy-, Enforcement-Layer.



\subsection{Linear Temporal Logic}
\label{ltl}

\subsection{Lattice-Based Access Control}
\label{lbac}
Eine wichtige Klasse von Zugriffskontrollmodellen sind \emph{Lattice-Based Access Control (LBAC)} Modelle.
Die Grundlagen von LBAC wurden in den 1970ern von Bell, LaPadula, Biba und Denning gelegt und entstanden 
aus den Bedürfnissen des Verteidigungssektors. Eine Übersicht über die verschiedenen Modelle 
und formale Definitionen, sowie weiterführende Referenzen finden sich bei Sandhu \cite{lbac2}.

LBAC Modelle konzentrieren sich auf die Betrachtung des Informationsflusses zwischen Sicherheitsklassen eines Systems.
Dazu wird jedem \emph{Objekt} des Systems eine \emph{Sicherheitsklasse} zugewiesen. 
Ein \emph{Objekt} sei hier informell definiert als ein Container für Informationen.
Wenn nun die Informationen eines Objekts \emph{x} zu einem Objekt \emph{y} fließen geht 
damit ein Informationsfluss einher. Dieser Informationsfluss findet zwischen den 
Sicherheitsklassen der jeweiligen Objekte statt.

Die \emph{Richtlinien} mit denen die Informationsflüsse beschrieben werden, können mithilfe von \emph{Lattices} beschrieben werden.
Ein \emph{Lattice} ist in diesem Zusammenhang die partiell geordnete Menge der Sicherheitsklassen. 
Im Modell von Bell und LaPadula werden beispielsweise die Objekte ihren Sicherheitsklassen zugewiesen.
Benutzer könne dann \emph{Subjekte}, wie zum Beispiel einen Prozess, erstellen, die eine Sicherheitsklasse besitzen, 
die von der Sicherheitsklasse des Benutzers im Lattice dominiert wird. 
Ein Subjekt kann genau dann Objekte lesen, wenn die Sicherheitsklasse des Objekts von der des Subjekts dominiert wird.
Auf ein Objekt schreibend zugreifen kann ein Subjekt genau dann, wenn die Sicherheitsklasse des Objekts die Sicherheitsklasse des Subjekts dominiert.

\subsection{Domain and Type Enforcement}
\label{dte}
Die Idee des \emph{Domain and Type Enforcement (DTE)}, von Badger et al. vorgeschlagen \cite{dte}, ist eine Erweiterung der Lattice-Based Access Control um \emph{Domänen} und \emph{Typen}.
Eine \emph{Domäne} enthält Subjekte und Objekte werden zu \emph{Typen} zugeordnet. 
Um den Informationsfluss zu kontrollieren, wird eine Matrix benutzt, 
in der die Lese- und Schreibrechte für jede Kombination von Domäne und Typ festgelegt werden.
Diese Art der Zugriffskontrolle ist beispielsweise dann anwendbar, 
wenn es gewünscht ist, vertrauenswürdige Pipelines zu etablieren. 
Wenn zum Beispiel gewünscht wird, dass Informationen nur über ein Subjekt einer 
dritten Domäne zum Zielobjekt fließen dürfen, ist dies mit klassischen LBAC-Modellen nicht möglich.
Dies liegt an der transitiven Natur der zugrundeliegenden Relation in LBAC. 

Da die im Lattice höher angesiedelten Sicherheitsklassen in LBAC alle unter ihnen 
liegenden Sicherheitsklassen dominieren und diese Relation transitiv ist, kann Information immer 
von allen dominierten Sicherheitsklassen direkt in die dominierende Sicherheitsklasse fließen.
Da bei DTE eine Matrix den zulässigen Informationsfluss definiert, 
ist diese Relation nicht länger transitiv und die Informationen können nach Wunsch über verschiedene Wege fließen.

\subsection{Role-Based Access Control}
\label{rbac}
\emph{Role-Based Access Control (RBAC)} ist ein Zugriffskontrollmodel, dass von Sandhu et al. und Ferraiolo et al. vorgeschlagen wurde \cite{sandhu96, ferraiolo2001proposed}.
Ein Benutzer sei im Folgenden, ohne Beschränkung der Allgemeinheit, definiert als ein menschlicher Benutzer.
Das Hauptmerkmal von RBAC sind \emph{Rollen}. Eine Rolle basiert auf der Idee eine Position innerhalb einer Organisation darzustellen, 
die mit bestimmten Rechten und Verantwortlichkeiten einhergeht. 
Daher werden jeder Rolle bestimmte \emph{Zugriffsrechte} auf \emph{Objekte} zugewiesen, wobei ein \emph{Objekt} Informationen bereitstellen oder erhalten kann.
Ein \emph{Zugriffsrecht} gewährt einer Rolle die Möglichkeit eine oder mehrere \emph{Operationen} auf einem oder mehreren Objekten auszuführen.

Ein Benutzer kann in RBAC mehreren Rollen zugehören und eine Rolle kann ebenfalls mehreren Benutzern zugeordnet sein.
Diese Relation wird als \emph{User Assignment (UA)} bezeichnet. 
Genau so können auch Zugriffsrechte zu Rollen in einer many-to-many Beziehung zugewiesen werden, 
die als \emph{Permission Assignment (PA)} bezeichnet wird.

Ein weiteres Konzept das in RBAC benutzt wird, sind \emph{Sessions}. 
Ein Benutzer kann einer oder mehreren Sessions zugeordnet sein und jede Session ist genau einem Benutzer zugeordnet.
In jeder Session kann wiederum eine Untermenge der Rollen, denen der Benutzer angehört, aktiviert sein.
Welche Rollen, und damit auch welche Zugriffsrechte, in einer Session aktiviert sind, kann durch die Funtktion \emph{session_roles} abgefragt werden.
Um zu erfahren, welche Sessions zu einem Benutzer gehören, kann die Funktion \emph{user_sessions} aufgerufen werden.

Die obigen Eigenschaften beschreiben \emph{Core RBAC} beziehungsweise \emph{$RBAC_0$}. Formal kann damit Core RBAC wie folgt definiert werden.
\begin{definition} [Core RBAC]
\begin{itemize}
\item \emph{USERS, ROLES, OPS und OBS}: Mengen der Benutzer, Rollen, Operationen und Objekten.
\item \emph{$UA \subseteq USERS \times ROLES$}: Menge der Zuweisungen von Benutzern zu Rollen.
\item \emph{$assigned\_users: (r: ROLES) \rightarrow 2^{USERS}, assigned\_users(r) = \{u \in USERS | (u, r) \in UA\}$}: Zuordnung der Rolle \emph{r} zu der zugehörigen Menge der Benutzer.
\item \emph{$PRMS = 2 ^ {(OPS \times OBS)}$}: Menge der Zugriffsrechte.
\item \emph{$PA \subseteq PRMS \times ROLES$}: Zuordnung der Zugriffsrechte zu Rollen.
\item \emph{$assigned\_permissions(r: ROLES) \rightarrow 2^{PRMS}, assigned\_permissions(r) = \{u \in USERS | (u, r) \in PA\}$}: Zuordnung der Rolle \emph{r} zu der zugehörigen Menge der Zugriffsrechte.
\item \emph{$Ob(p: PRMS) \rightarrow \{op \subseteq OBS\}$}: Gibt die Menge der Operationen an, die einem Zugriffsrecht \emph{p} zugeordnet sind.
\item \emph{$Ob(p: PRMS) \rightarrow \{ob \subseteq OBS\}$}: Gibt die Menge der Objekte an, die einem Zugriffsrecht \emph{p} zugeordnet sind.
\item \emph{SESSIONS}: Menge der Sessions.
\item \emph{$user\_sessions(u: Users) \rightarrow 2^{SESSIONS}$}: Zuordnung eines Benutzers \emph{u} zu einer Menge von Sessions.
\item \emph{$session\_roles(s: SESSIONS) \rightarrow 2^{ROLES}, session\_roles(s_i) \subseteq \{r \in ROLES | (session\_users(si), r) \in UA\}$}: Zuordnung einer Session \emph{s} zu der zugehörigen Menge von Rollen.
\item \emph{$avail\_session\_perms(s: SESSIONS) \rightarrow 2^{PRMS}$, \bigcup{r \in sesseion\_roles(s)}{assigned\permissions(r)}}: Die Zugriffsrechte, die einem Benutzer in einer Session zur Verfügung stehen.
\end{itemize}
\end{definition}

Das Core RBAC Modell definiert die grundlegenden Eigenschaften jedes RBAC Modells. 
Es wurden auch weitergehende Modelle beschrieben, wie hierarchisches RBAC und 
constrained RBAC. Für die Definitionen dieser Modelle sei an dieser Stelle jedoch lediglich auf Ferraiolo et al. \cite{ferraiolo2001proposed} verwiesen

TODO: Übergang zum nächsten Kapitel

\section{G-SIS}
\label{gsis}


\subsection{Klassifizierung von g-SIS Modellen}
In vielen Fällen ist es notwendig, die Benutzer eines abzusichernden Systems in mehr als eine Gruppe einzuteilen. 
Diese Gruppen können sowohl nach der Art der Beziehung zwischen den einzelnen Gruppen unterschieden werden,
als auch nach der Art der Beziehungen zwischen den Benutzern innerhalb einer Gruppe.

Verschiedene Gruppen in g-SIS könenn \emph{verbunden} oder \emph{isoliert} sein.
Sind Gruppen untereinander \emph{isoliert}, hat die Mitgliedschaft eines Benutzers, 
Subjekts oder Objekts keinen Einfluss auf die Mitgliedschaft eines Benutzers, 
Subjekts oder Objekts in einer anderen Gruppe.

Im Falle mehrerer \emph{verbundener} Gruppen kann die Mitgliedschaft in einer Gruppe 
verschiedene Auswirkungen auf die Mitgliedschaft und die möglichen Operationen in einer anderen Gruppe haben.

Die Mitglieder innerhalb einer Gruppe können wiederum alle die gleichen Zugriffsrechte besitzen.
Diese Art der Gruppe wird in g-SIS definiert als eine \emph{undifferenzierte} Gruppe.
Im Gegensatz dazu können in einer \emph{differenzierten} Gruppe verschiedenen Benutzern 
auch verschiedene Zugriffsrechte eingeräumt werden.

Die genannten Unterschiede und ihre Kombinationen führen zu vier möglichen g-SIS Modellen.
Diese sind das \emph{isoliert undifferenzierte Modell}, das \emph{isoliert differenzierte Modell}, das \emph{verbunden undifferenzierte Modell} 
sowie das \emph{verbunden differenzierte Modell}.

Das \emph{isoliert undifferenzierte Modell} beschreibt das Modell, 
in dem die Mitgliedschaft in einer Gruppe keinen Einfluss auf andere Gruppen im System hat.
Ebenso gibt es innerhalb der einzelnen Gruppen keinen Unterschied in den Zugriffsrechten der einzelnen Mitglieder,
außer den Unterschieden, die durch die g-SIS Core Properties (s.h. Abschnitt \ref{properties}) 
für die Zugriffsrechte festgelegt werden. 
Das isolierte undifferenzierte Modell bildet damit die Grundlage für die anderen g-SIS Modelle.

Im \emph{verbundenen undifferenzierten Modell} kann die Mitgliedschaft in einer Gruppe 
verschiedene Auswirkungen auf andere Gruppen haben, die im folgenden Abschnitt genauer betrachtet werden.
Desweiteren haben die Mitglieder untereinander, wie im isolierten undifferenzierten Modell, die selben Zugriffsrechte.
In Abbildung \ref{img_models} sind die Beziehungen zwischen den einzelnen 
g-SIS Modellen grafisch dargestellt. Das isoliert undifferenzierte Modell ist in allen anderen Modellen enthalten.
Das isoliert differenzierte Modell und das verbunden undifferenzierte Modell sind in diesem Sinne nicht vergleichbar.
Das verbunden differenzierte Modell wiederum enthält alle anderen Modelle. 

Im Folgenden soll allerdings lediglich das verbunden undifferenzierte Modell betrachtet werden, 
da dies ausreicht um Aussagen über die Beziehungen zwischen einzelnen Gruppen machen zu können.

TODO: Grafik einbauen
\begin{figure}[htb] 
\centerline{\includegraphics*[width=0.8\textwidth]{an_example_picture}}
\caption{Hier g-SIS Modelle} 
\label{img_models}
\end{figure}

\subsection{Beziehungen zwischen Gruppen}

Inter-group Relationship Semantics für connected undifferentiated group models.

\section{Andere Policies in g-SIS}
\label{other_policies}

LBAC Policies in g-SIS, Domain and Type Enforcement.

RBAC Policies in g-SIS.

\section{Fazit}
\label{summary}


% \label{tmp_Subsection}
% Dies ist ein Unterabschnitt. Mit \LaTeX\ kann man auch prima Tabellen (siehe
% Beispieltabelle~\ref{tmp_tabelle}) erzeugen:
% \begin{table}[!htb] % das ! vor dem h heißt, dass LaTeX versucht, das "Gleitobjekt" 
                    % % wirklich hier zu plazieren
% \begin{center} % Zentrieren der Tabelle
% \begin{tabular}{|l||c|r|p{6.5cm}|}
% \hline
% \multicolumn{4}{|c|}{Beispieltabelle} \\
% \hline
% \hline
% Spalte 1       &    Spalte 2     &      Spalte 3     &        Spalte4 \\
% \hline
% linksbündig    & zentriert       & rechtsbündig      & 6.5cm breit, Fließtext
                                                       % ist möglich, d.h. der Text
                                                       % wird automatisch umgebrochen,
                                                       % allerdings im Blocksatz gesetzt.\\
% \hline
               % & \multicolumn{2}{c|}
                 % {Ein Eintrag über Spalte 2--3}      & Es gibt auch die Möglichkeit
                                                       % Flattersatz in Spalten einzusetzen.
                                                       % Das eignet sich besonders für kurze
                                                       % Spalten. Auch grau unterlegte 
                                                       % Tabellenzeilen oder Tabellen über
                                                       % mehrere Seiten sind möglich.\\
% \hline
% Man            &                 &                   &                          \\
% kann auch      & horizontale     & Linien            & einfach weglassen. \\
% \hline
% \end{tabular}
% \end{center}
% \caption{Beispiel einer Tabelle\label{tmp_tabelle}}
% \end{table}
%%
% Für Flattersatz in Tabellenspalten:
% \newcolumntype{z}{>{\PBS{\raggedright\hspace{0pt}}}p{5cm}}
% statt p{5cm} einfach z verwenden, z.B. \begin{tabular}{|l||c|r|z|}
% Für graue Tabellenspalten (z.B. zentrierter Text):
% \newcolumntype{g}{>{\columncolor[gray]{0.8}}c} % grau
% \newcolumntype{G}{>{\columncolor[gray]{0.9}}c} % helleres grau
% \begin{tabular}{|l||c|G|g|} oder auch \multicolumn{4}{|g|}{Eine ganze Zeile}
% Aber Achtung: Im xdvi werden graue Zeilen nicht korrekt dargestellt, erst
% im PostScript- oder PDF-Dokument bzw. Ghostview.
%
% /\ /\ /\ /\ /\ /\ /\ /\ /\ /\ /\ /\ /\ /\ /\ /\ /\ /\ /\ /\ /\ /\ /\ /\ /\ 
%
% --> Üblicherweise wird nur die Literatur aufgelistet, die auch referenziert
%     wird. Möchte man auch nichtreferenzierte Literatur einschließen, so
%     kann man dies mit \nocite{<citelabel>} tun.
%\nocite{*}
%     In die folgende Zeile sollte die benötigte Literaturdatenbankdatei
%     eingetragen werden (im Normalfall nicht zu ändern):

% \/ \/ \/ \/ \/ \/ \/ \/ \/ \/ \/ \/ \/ \/ \/ \/ \/ \/ \/ \/ \/ \/ \/ \/ \/ 
% Sollte noch zusätzliche Literatur verwendet worden sein, dann bitte die 
% folgende Zeile auskommentieren.
\bibliography{my_bibliography}
% /\ /\ /\ /\ /\ /\ /\ /\ /\ /\ /\ /\ /\ /\ /\ /\ /\ /\ /\ /\ /\ /\ /\ /\ /\ 
%
\docend
%%% end of document
